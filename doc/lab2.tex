\documentclass[a4paper, 12pt]{article}
\usepackage{cmap}
\usepackage[12pt]{extsizes}			
\usepackage{mathtext} 				
\usepackage[T2A]{fontenc}			
\usepackage[utf8]{inputenc}			
\usepackage[english,russian]{babel}
\usepackage{setspace}
\singlespacing
\usepackage{amsmath,amsfonts,amssymb,amsthm,mathtools}
\usepackage{fancyhdr}
\usepackage{soulutf8}
\usepackage{euscript}
\usepackage{mathrsfs}
\usepackage{listings}
\pagestyle{fancy}
\usepackage{indentfirst}
\usepackage[top=10mm]{geometry}
\rhead{}
\lhead{}
\renewcommand{\headrulewidth}{0mm}
\usepackage{tocloft}
\renewcommand{\cftsecleader}{\cftdotfill{\cftdotsep}}
\usepackage[dvipsnames]{xcolor}

\lstdefinestyle{mystyle}{ 
	keywordstyle=\color{OliveGreen},
	numberstyle=\tiny\color{Gray},
	stringstyle=\color{BurntOrange},
	basicstyle=\ttfamily\footnotesize,
	breakatwhitespace=false,         
	breaklines=true,                 
	captionpos=b,                    
	keepspaces=true,                 
	numbers=left,                    
	numbersep=5pt,                  
	showspaces=false,                
	showstringspaces=false,
	showtabs=false,                  
	tabsize=2
}

\lstset{style=mystyle}

\begin{document}
\thispagestyle{empty}	
\begin{center}
	Московский авиационный институт
	
	(Национальный исследовательский университет)
	
	Факультет "Информационные технологии и прикладная математика"
	
	Кафедра "Вычислительная математика и программирование"
	
\end{center}
\vspace{40ex}
\begin{center}
	\textbf{\large{Лабораторная работа №2 по курсу\linebreak \textquotedblleft Операционные системы\textquotedblright}}
\end{center}
\vspace{35ex}
\begin{flushright}
	\textit{Студент: } Живалев Е.А.
	
	\vspace{2ex}
	\textit{Группа: } М8О-206Б
	
	\vspace{2ex}
	\textit{Преподаватель: } Соколов А.А.
	
	\vspace{2ex}
	\textit{Вариант: } 27
	
	\vspace{2ex}
	\textit{Оценка: } \underline{\quad\quad\quad\quad\quad\quad}
	
	 \vspace{2ex}
	\textit{Дата: } \underline{\quad\quad\quad\quad\quad\quad}
	
	\vspace{2ex}
	\textit{Подпись: } \underline{\quad\quad\quad\quad\quad\quad}
	
\end{flushright}

\vspace{5ex}

\begin{vfill}
	\begin{center}
		Москва, 2019
	\end{center}	
\end{vfill}
\newpage

\section{Задание}

Программе на вход поступает число n - количество процессов, которое будет создано в ходе выполнения и программы, и n названий файлов, в которые эти процессы будут записывать данные. Родительский процесс создает n дочерних процессов, которым поочередно передает символы с входной строки. Дочерний процесс записывает полученные символ в переданный ему файл.

В ходе выполнения лабораторной работы были использованы следующие системные вызовы:

\begin{itemize}
	\item read - использовался для чтения данных с входной строки
	\item write - использовался для записи данных в файлы
	\item fork - использовался для создания дочерних процессов
	\item pipe - использовался для создания однонаправленных, использованных для передачи родительским процессом символов дочерним процессам
\end{itemize}
\section{Описание работы программы}

С помощью функции readLine, которая, используя read считывает одну строку с входной строки, программа получает количество процессов(число n), которые необходимо создать и имена файлов, в которые должна производиться запись. Затем создаётся n пайпов и n процессов. Затем в родительском процессе в цикле с условием, что возвращенное функцией read число больше 0 происходит запись считанного символа в определенный пайп. В то же время дочерний процесс считывает из пайпа символ и записывает его в свой файл.

\newpage

\section{Исходный код}

\textbf{\large{main.c}}
\lstinputlisting[language=C]{main.c}
\newpage
\section{Консоль}

\begin{verbatim}
	qelderdelta@qelderdelta-UX331UA:~/Study/OS/os_lab_2/src$ ./a.out
	3
	1.txt
	2.txt
	3.txt
	ttthhhiiisss   ssshhhooouuulllddd   bbbeee   iiinnn   fffiiillleee123                            
	qelderdelta@qelderdelta-UX331UA:~/Study/OS/os_lab_2/src$ cat 1.txt
	this should be in file1
	qelderdelta@qelderdelta-UX331UA:~/Study/OS/os_lab_2/src$ cat 2.txt
	this should be in file2
	qelderdelta@qelderdelta-UX331UA:~/Study/OS/os_lab_2/src$ cat 3.txt
	this should be in file3
\end{verbatim}
\newpage
\section{Выводы}

В ходе выполнения лабораторной работы я познакомился с некоторым системными вызовами, например, fork, pipe, а также увидел как их можно применять на практике. Также я более подробно изучил как устроены процессы и пайпы в ОС Linux.
\end{document}